\documentclass[a4paper,11pt]{article}

\usepackage{graphicx}
\usepackage{subfig}


\begin{document}
\title{ML CS726 Fall'11 Project 2 Complex Classifiers}
\author{Angjoo Kanazawa, Ran Liu, Austin Myers}
\maketitle


\section{Gradient Descent and Linear Classification}
\subsection{WU1}
\textsf{Find a few values of step size where it converges and a few
  values where it diverges. Where does the threshold seem to be?}\\
It depends on what the number of iteration is, but with 100
iterations, from step 0.1 to 6.5 it finds a solution close to
0. But right after 6.6, it starts to diverge and for any value after
6.7 it diverges. The treshold seems to be around 6.6.

\subsection{WU2}
\textsf{Come up with a non-convex univariate optimization
  problem. Plot the function you're trying to minimize and show two
  runs of gd, one where it gets caught in a local minimum and one
  where it manages to make it to a global minimum. (Use different
  starting points to accomplish this.)}\\
Using $f(x) = sin(\pi x) + x^2/2$, $f'(x) = \pi cos(\pi x) + x$, and
10 iterations, the
global minimum happens at $x\approx -0.45385$.
If we start at 0, we can find the global minimum, but if we start at
1, gd gets caught in a local minimum $1.357$
\begin{verbatim}
>>> gd.gd(f, derF, 0, 10, 0.2)
(-0.45385351939658519, array([ 0.        , -0.72244765, -0.84494752, -0.88472033, -0.88650671,
       -0.88651822, -0.88651823, -0.88651823, -0.88651823, -0.88651823,
       -0.88651823]))
>>> gd.gd(f, derF, 1, 10, 0.2)
(1.3577434052579487, array([  1.22464680e-16,   4.52961014e-02,   2.50055080e-02,
         2.00175811e-02,   1.99477186e-02,   1.99477147e-02,
         1.99477147e-02,   1.99477147e-02,   1.99477147e-02,
         1.99477147e-02,   1.99477147e-02]))
\end{verbatim}

\subsection{WU3}
\textsf{Why does the logistic classifier produce much larger weights
  than the others, even though they all get basically the same
  classification performance?}\\ Don't have to answer this.

\section{Warm Up with ML Tools}
\label{sec:warmup}
\subsection{WU4}
\textsf{What are the five features with largest positive weight and what
are the five features with largest negative weight? Do these seem "right"
based on the task?}\\

\begin{enumerate}
 \item motif -1.21427941322326660156
 \item window -1.15418136119842529297
 \item server -0.94833439588546752930
 \item list -0.89077484607696533203
 \item x -0.86568439006805419922
\end{enumerate}

\begin{enumerate}
 \item graphics 1.09135174751281738281
 \item images 0.72243607044219970703
 \item image 0.72001230716705322266
 \item card 0.71239823102951049805
 \item xx 0.69124054908752441406
\end{enumerate}

\subsection{WU5}
\textsf{ Draw the tree. How do the selected features compare to the
  features from the logistic regression model? Which features seem
  "better" and why? If you use a depth 10 tree, how well do you do on
  test data?}\\
TODO: add the image of the tree later
The selected features include four of the best/worst features from
the logistic regression (``graphics'', ``motif'', ``window'', and ``x'').

Features can be considered ``better'' when the ratio bewteen class 1
and class 0 at the leaf is higher and when the frequency of the number
of instances that got to that leaf is high. In this respect, ``motif''
and ``vga'' are the better features. ``motif'' has high frequency
(total 583 instances fall have no motif) and high ratio (about 80\% of
those without ``motif'' are in class 1).  395 instances fall in
``vga'', and 84\% of those with no ``vga'' are in class 0. If they
have ``vga'', 100\% of them are in class 1 (although this may not be the
best indicator because only 13 data falls into yes ``vga''.) 
In this respect, ``usr'' is one of the worst features in this tree
because the relative frequency is low, adn when there is no ``usr'',
the ratio of class1 to class0 is 6:4.

If I use a depth 10 tree, I get a test error of 20.53\%, (0.463\%
improvement).

\subsection{WU6}
\textsf{}\\

\subsection{WU7}
\textsf{}\\

\section{Reductions for Multiclass Classification}
\subsection{WU8}
\textsf{}\\

\subsection{WU9}
\textsf{}\\

\subsection{Ranking or Collective Classification}
\subsection{WU10a}
\textsf{}\\


\subsection{WU10b}
\textsf{}\\



      
\end{document}
